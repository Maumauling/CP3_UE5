\documentclass[%
	paper=A4,	% stellt auf A4-Papier
	pagesize,	% gibt Papiergröße weiter
	DIV=calc,	% errechnet Satzspiegel
	smallheadings,	% kleinere Überschriften
	ngerman		% neue Rechtschreibung
]{scrartcl}
\usepackage{BenMathTemplate}
\usepackage{BenTextTemplate}
\definecolor{red}{rgb}{255,0,0}
\definecolor{black}{rgb}{0,0,0}
\newcommand{\mycbox}[1]{\tikz{\path[draw=#1] (0,0) rectangle (8pt,8pt);}}

\title{{\bf Wissenschaftliches Rechnen III / CP III}\\Übungsblatt 5}
\author{Tizia Kaplan (545978)\\Benjamin Dummer (532716)\\Antoine Hoffmann (426675)\\Gruppe 10}
\date{01.06.2016}

\begin{document}
\maketitle
Online-Version: \href{https://www.github.com/BeDummer/CP3_UE5}{\url{https://www.github.com/BeDummer/CP3_UE5}}

\section*{Aufgabe 5.1}

\section*{Aufgabe 5.2}
Es gilt das \emph{Amdahlsche Gesetz} $S_p(N) \leq \frac{1}{f} $ zu zeigen. Es wird für jede Rechenoperation die Zeit $\Delta_T$ angenommen.

\begin{eqnarray}
	T_s (N) & = & N \Delta_T \\
	T_p (N) & = & f N \Delta_T + \dfrac{(1-f)N \Delta_T}{p} \\
	S_p (N) & = & \dfrac{T_s(N)}{T_p(s)} \\
		& = & \dfrac{N \Delta_T}{f N \Delta_T + \frac{(1-f) N \Delta_T}{p}} \nonumber \\
		& = & \dfrac{1}{f+\frac{1-f}{p}} = \dfrac{p}{f (p-1) +\color{red} 1} \nonumber \\
	\mbox{da } f < 1, & \leq & \dfrac{p}{f (p-1) +\color{red} f} = \dfrac{p}{f p} = \dfrac{1}{f} \quad \quad \mycbox{black} \nonumber
\end{eqnarray}
Von der vorletzten zur letzten Zeile wird im Nenner $f-1$ addiert. Da $f<1$ gilt, wird der Nenner durch diesen Schritt kleiner und somit der gesamte Term größer.

\section*{Anhänge}
\begin{itemize}
	\item Datei: \url{bla.cu} (Hauptprogramm)
\end{itemize}
\end{document}


%% Beispiel fuer Einbindung eines Bildes

%\begin{figure}
%  \centering
%  \includegraphics[width=.75\textwidth]{Dateiname}
%  \caption{Beschriftung}
%\end{figure}


%% Beispiel fuer Tabelle im Mathe-Modus

%\begin{eqnarray} \nonumber
%	\begin{array}{l|c|c}
% \mbox{Variablentyp} & \mbox{\tt int} & \mbox{\tt double}\\ \hline
% \mbox{Laufzeit [ms]} & 1.47 & 2.01  \\ 
% \mbox{Efficiency [\%]} & 100 & 100 \\
% \mbox{Throughput [GB/s]} & 47.8 & 69.4  \\
% \mbox{Occupancy} & 0.9985 & 0.9992
%	\end{array}
%\end{eqnarray}
